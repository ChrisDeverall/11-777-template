% File project.tex
%% Style files for ACL 2021
\documentclass[11pt,a4paper]{article}
\usepackage[hyperref]{acl2021}
\usepackage{times}
\usepackage{booktabs}
\usepackage{todonotes}
\usepackage{latexsym}
\renewcommand{\UrlFont}{\ttfamily\small}

% This is not strictly necessary, and may be commented out,
% but it will improve the layout of the manuscript,
% and will typically save some space.
\usepackage{microtype}

\aclfinalcopy 

\newcommand\BibTeX{B\textsc{ib}\TeX}

\title{11-777 Spring 2021 Class Project}

\author{
  First Last 1\thanks{\hspace{4pt}Everyone Contributed Equally -- Alphabetical order} \hspace{2em} First Last 2$^*$ \hspace{2em} First Last 3$^*$ \hspace{2em} First Last 4$^*$ \\
  \texttt{\{ID1, ID2, ID3, ID4\}@andrew.cmu.edu}
  }

\date{}

\begin{document}
\maketitle
\begin{abstract}
Template for 11-777 Reports using the ACL 2021 Style File 
\end{abstract}

\section{Introduction and Problem Definition (1-1.25 pages)}
\textbf{Thesis statement or Hypothesis we are aiming to prove}\\
``Our approach is better is not a hypothesis"
\begin{figure}
\missingfigure[figwidth=\linewidth]{This is a simple example/demonstration figure that explains your task and insight}
\end{figure}

\clearpage
\section{Related Work and Background (1-1.5 pages)}

\subsection{Visual Language Navigation} 



\paragraph{Literature 1}
\paragraph{Literature 2}
\paragraph{Literature 3}
\paragraph{Literature 4}

\clearpage
\section{Task Definitions and Data Analysis (1 page)}
\subsection{Task formulation}

The high-level task for the ALFRED dataset is to predict the correct sequence of household actions given textual instructions and visual feedback. The ALFRED dataset is unique in that the actions are long, compositional and non-reversible. An example of this is that the agent must pick up a knife in order to slice a potato. Moreover, once a potato is sliced, it cannot be put back together. Actions consist of navigation within a room or interaction with an object. For interaction, a pixelwise mask must be predicted whereby the object with the highest intersection-over-union score with the interaction mask is acted upon. The textual instructions come in the form of both high-level task descriptions and step by step commands. The visual feedback is ego-centric and the next image is provided after each action is performed. Compared to previous datasets, there is a diverse range of tasks. Specifically there are 7 high-level task types parameterized by a combination of 84 object classes and 120 household scenes.

\subsection{Data Analysis}

\subsubsection{Action Space}

As can be seen in the table below, each task is composed by a large number of low level actions with the largest tasks consisting of 16 actions. THIS IS FOR THE VAL SEEN SET 

\begin{tabular}{lr}
\toprule
{} &  Count of Actions per task \\
\midrule
mean &          6.47 \\
std  &          2.62 \\
max  &         16.00 \\
min  &          3.00 \\
\bottomrule
\end{tabular}


\subsubsection{Hello}


\clearpage
\section{Models (2 pages)}

\subsection{Baselines}
Both existing baselines explained with citations and novel ones missing from the current literature

\subsection{Proposed Approach}

\clearpage
\begin{table*}[t]
\centering
\begin{tabular}{lrrrr}
\toprule
                            & \multicolumn{2}{c}{Dev} & \multicolumn{2}{c}{Test}\\
Methods                     & Accuracy $\uparrow$ & $L_2$ Error $\downarrow$ & Accuracy $\uparrow$ & $L_2$ Error $\downarrow$ \\
\midrule
Previous Approach 1 \cite{} & & & & \\
Previous Approach 2 \cite{} & & & & \\
Previous Approach 3 \cite{} & & & & \\
\midrule
Proposed Method             & & & & \\
\bottomrule
\end{tabular}
\end{table*}
\section{Results (1 page)}
The columns above are just examples that should be expanded to include all metrics and baselines.

\clearpage
\section{Analysis (2 pages)}
This section should include at least two to three plots
\subsection{Ablations and Their Implications}

\subsection{Qualitative Analysis and Examples}
This section should likely contain a table of examples demonstrating how the current approach succeeds/fails.

% Please use 
\bibliographystyle{acl_natbib}
\bibliography{references}

%\appendix



\end{document}
